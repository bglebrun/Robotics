%%%%%%%%%%%%%%%%%%%%%%%%%%%%%%%%%%%%%%%%%
% Memo
% LaTeX Template
% Version 1.0 (30/12/13)
%
% This template has been downloaded from:
% http://www.LaTeXTemplates.com
%
% Original author:
% Rob Oakes (http://www.oak-tree.us) with modifications by:
% Vel (vel@latextemplates.com)
%
% License:
% CC BY-NC-SA 3.0 (http://creativecommons.org/licenses/by-nc-sa/3.0/)
%
%%%%%%%%%%%%%%%%%%%%%%%%%%%%%%%%%%%%%%%%%

\documentclass[letterpaper,11pt]{texMemo} % Set the paper size (letterpaper, a4paper, etc) and font size (10pt, 11pt or 12pt)

\usepackage{parskip} % Adds spacing between paragraphs
\usepackage[colorlinks]{hyperref}
\usepackage{graphicx}
\usepackage{float}
\usepackage{listings}
\hypersetup{citecolor=DeepPink4}
\hypersetup{linkcolor=red}
\hypersetup{urlcolor=blue}
\usepackage{cleveref}
\setlength{\parindent}{15pt} % Indent paragraphs

%----------------------------------------------------------------------------------------
%	MEMO INFORMATION
%----------------------------------------------------------------------------------------

%----------------------------------------------------------------------------------------
%	MEMO INFORMATION
%----------------------------------------------------------------------------------------

\memoto{Dr.Jeff McGough} % Recipient(s)

\memofrom{Benjamin Lebrun} % Sender(s)

\memosubject{Homework 1} % Memo subject

\memodate{\today} % Date, set to \today for automatically printing todays date

%\logo{\includegraphics[width=0.1\textwidth]{logo.png}} % Institution logo at the top right of the memo, comment out this line for no logo

%----------------------------------------------------------------------------------------

\begin{document}


\maketitle % Print the memo header information

%----------------------------------------------------------------------------------------
%	MEMO CONTENT
%----------------------------------------------------------------------------------------

\section*{Problem 2.10}
\subsection*{Problem statement}
Assume that you have a two link manipulator with $a_1=15cm$ and $a_2=15cm$ and that the base of the manipulator is at
the origin of the coordinate system. Write a Python program to take the list of workspace points and plug them into the
inverse kinematics formulas for the two link manipulator. Plot these points on a graph where $\theta_1$ is the horizontal
axis and $\theta_2$ is the vertical axis. You will have to adjust some aspects to get a good looking plot.
(Scale factors etc.) Test your code on the workspace line (a) $x+y=25, x,y>0$ and (b) $x=10cos(t)+15, y=10sin(t)$ for $0≤t≤\pi$.
The point here is to see what the configuration space curve looks like.

\subsection*{Solution approach and algorithm description.}
For this problem, we can use the inverse kinematic equations 
\[
    D = \frac{x^2+y^2-a$^2_1$-a$^2_2$}{2a_1 a_2}
\]

\end{document}