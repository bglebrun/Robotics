%%%%%%%%%%%%%%%%%%%%%%%%%%%%%%%%%%%%%%%%%
% Memo
% LaTeX Template
% Version 1.0 (30/12/13)
%
% This template has been downloaded from:
% http://www.LaTeXTemplates.com
%
% Original author:
% Rob Oakes (http://www.oak-tree.us) with modifications by:
% Vel (vel@latextemplates.com)
%
% License:
% CC BY-NC-SA 3.0 (http://creativecommons.org/licenses/by-nc-sa/3.0/)
%
%%%%%%%%%%%%%%%%%%%%%%%%%%%%%%%%%%%%%%%%%

\documentclass[letterpaper,11pt]{texMemo} % Set the paper size (letterpaper, a4paper, etc) and font size (10pt, 11pt or 12pt)

\usepackage{parskip} % Adds spacing between paragraphs
\usepackage[colorlinks]{hyperref}
\usepackage{graphicx}
\usepackage{gensymb}
\usepackage{float}
\usepackage{listings}
\usepackage{siunitx}
\usepackage{csvsimple}
\hypersetup{citecolor=DeepPink4}
\hypersetup{linkcolor=red}
\hypersetup{urlcolor=blue}
\usepackage{cleveref}
\setlength{\parindent}{15pt} % Indent paragraphs

%----------------------------------------------------------------------------------------
%	MEMO INFORMATION
%----------------------------------------------------------------------------------------

%----------------------------------------------------------------------------------------
%	MEMO INFORMATION
%----------------------------------------------------------------------------------------

\memoto{Dr.Jeff McGough} % Recipient(s)

\memofrom{Benjamin Lebrun} % Sender(s)

\memosubject{Homework 3} % Memo subject

\memodate{\today} % Date, set to \today for automatically printing todays date

%\logo{\includegraphics[width=0.1\textwidth]{logo.png}} % Institution logo at the top right of the memo, comment out this line for no logo

%----------------------------------------------------------------------------------------

\begin{document}


\maketitle % Print the memo header information

%----------------------------------------------------------------------------------------
%	MEMO CONTENT
%----------------------------------------------------------------------------------------

\section*{Problem 8.1}
\subsection*{Problem statement}
Assume that you are working in a large event center which has beacons located around the 
facility. Estimate the location of a robot, $(a,b,c)$, if the $(x,y,z)$ location of the beacon 
and the distance from the beacon to the robot, $d$, are given in the table below.

\csvautotabular{../P1.csv}

\subsection*{Solution approach and algorithm description.}

For this problem we know that the given coordinates and distances are just centers of spheres/circles with
the distance to the beacon as a radius. In the book we are given in Chapter 8, the equations to use for a
two dimensional plane using beacons. We will adapt this by adding a $z$ term to the equations to account for
an extra dimension of resolution. The given equations in 2D are

\[
    (x-a_i)^2 + (y-b_i)^2 = r_i^2 , \quad (x-a_j)^2 + (y-b_j)^2 = r_j^2 .
\]

For 3 Dimensions

\[
    (x-a_i)^2 + (y-b_i)^2 +(z-c_i)^2 = r_i^2 , \quad (x-a_j)^2 + (y-b_j)^2 + (z-b_j)^2 = r_j^2
\]

Expanded, and then rearranged in the book to allow us to find the difference between two beacons
we subtract one equation from the other.

\[
    2(a_j-a_i)x + 2(b_j-b_i)y + 2(c_j-b_i)z  = r_i^2 - r_j^2 - a_i^2 + a_j^2 - b_i^2 + b_j^2 - c_i^2 + c_j^2
\]

\[
    2(a_k-a_i)x + 2(b_k-b_i)y + 2(c_k-b_i)z = r_i^2 - r_k^2  - a_i^2 + a_k^2 - b_i^2  + b_k^2 - c_i^2 + c_j^2
\]

We repeat this for a third equation using $a_l$, $b_l$, and $c_l$ terms for 3 equations with 3 unknowns, allowing 
us to solve this as a system of linear equations. When we perform this calculation in Matlab, we find that 
the position of the robot sits at $(882.28, 444.15, 523.30)$ for position in $(x,y,z)$. Below you can find our Matlab
code.


\newpage
\subsection*{Code for Problem 5}
\begin{tiny}
\lstinputlisting{../P1.m}
\end{tiny}

\subsection*{Matlab output}
\begin{tiny}
\lstinputlisting{../solutionFile.txt}
\end{tiny}

\newpage
\section*{Problem 8.2}
\subsection*{Problem statement}
If you are using a laser diode to build a distance sensor, you need some method to determine the
travel time. Instead of using pulses and a clock, try using phase shifts. What is the wavelength 
of the modulated frequency of 10MHz? If you measure a 10 degree phase shift, this value corresponds 
to what distances? What if the phase shift measurement has noise: zero mean with standard deviation 
0.1? How does one get a good estimate of position if the ranges to be measured are from 20 meters 
to 250 meters?

\subsection*{Solution approach and algorithm description.}

To find our wavelength we use the following equation

\[
    \lambda = \frac{c}{v}
\]

Where $c$ is the speed of light, $\lambda$ is our wavelength, and $v$ is our frequency. Thus

\[
    \lambda = \frac{3*10^8 \frac{m}{s}}{10*10^6}
\]

We find that $\lambda = 30$ meters.

For a $10\degree$ phase shift, we use the equation for phase shift of the wavelength distance

\[
    D = \frac{\lambda}{4\pi} \theta
\]

$D = \frac{10}{360}30 = 0.8333$ meters.

For the full trip this would be $0.8333 + 30n$ for $n = 0,1,2,3\dots$
for some values of $n$, would then create the solutions $=0.8333, 30.8333, 60.8333,\dots$

For a phase shift measurement with noise, we pick a second frequency that isn't a factor of 
10 in the MHz range, then use the same calculations with a measured phase shift to find a second
set of points and correlate the answer from the point that the two share with our standard deviation 
factored into the calculation, in this case with any distance between the 20 and 250 meters 
range.


\newpage
\section*{Problem 9.1}
\subsection*{Problem statement}
Assume you have a laser triangulation system as shown in Fig. 9.2 given by (9.1) and that $f  = 8$mm,
$b=30$cm. What are the ranges for $\alpha$ and $u$ if we need to measure target distances in a region
(in cm) $20 < z < 100$ and $10<x<30$?

\subsection*{Solution approach and algorithm description.}

For the given range of $x$ and $z$, we know that the angle $\alpha$ can only reasonably be in the range
$0<\alpha<90\degree$, given this, we find the maximum and minimum for these angles with the equations

\[
    x = \frac{b u}{f\cot \alpha + u},  \quad
z = \frac{b f}{f\cot \alpha + u}
\]

with our given vales, these become

\[
    x = \frac{30 u}{0.8\cot \alpha + u},  \quad
z = \frac{30 * 0.8}{0.8\cot \alpha + u}
\]

With a calculator defining the limits to be $0<\alpha<90\degree$, $20 < z < 100$ and $10<x<30$. Our answers become 
$45\degree<\alpha<71.565\degree$ and $1.6\cot \alpha < u < -0.8(\cot \alpha - 3)$. For our range of $u$ we use the
maximum and minimum found for $\alpha$, which then defines $u$ along the range $0.53<u<2.1334$ cm.

\newpage
\section*{Problem 9.2}
\subsection*{Problem statement}
Assume you have two cameras that are calibrated into a stereo pair with a baseline of 10cm, and focal depth of 7mm. If the error is 10\% 
on $v_1$ and $v_2$, $v_1=2$mm and $v_2=3mm$, what is the error on the depth measurement $z$? Your answer shoudl be a percentage relative 
to the error free number. Hint: If $v_1=2$ then a 10\% error ranges from 1.8 to 2.2 [Although not required, another way to approach this problem is the total differential from calculus.]

\subsection*{Solution approach and algorithm description.}

For this problem we begin with the equation for $z$ given th baseline and focal depth, as well as values for $v_1$ and $v_2$.

\[
    z = \frac{fb}{v_1+v_2}
\]

For $z = \frac{0.7*10}{0.2+0.3} = 14$cm is our exact answer. However, we want values for $v_1$ and $v_2$  $\pm$10\%. This range then 
becomes $1.8\geq v_1 \geq 2.2$ and $2.7 \geq v_2 \geq 3.3$. For this, we wrote a simple python script 
that uses a numpy arange, combines the two terms $v_1$ and $v_2$ and then searches the results for local 
minima and maxima. Using this method, we found that the largest value for $z$ then becomes $12.7273 \geq z \geq 15.556$.

However, our problem asks us to present this data as a percent error. Using the percentage error equation
$\frac{|theoretical - experimental|}{|theoretical|} * 100\%$ with 14 as our theoretical value, the largest 
percentage is the upper bound, giving us a total percent error of 11.1\%.

\newpage
\subsection*{Code for Problem 4}
\begin{tiny}
\lstinputlisting{../P4.py}
\end{tiny}

\end{document}